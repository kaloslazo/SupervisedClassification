\documentclass[conference]{IEEEtran}
\IEEEoverridecommandlockouts

\usepackage{cite}
\usepackage{amsmath,amssymb,amsfonts}
\usepackage{algorithmic}
\usepackage{graphicx}
\usepackage{textcomp}
\usepackage{xcolor}

\def\BibTeX{{\rm B\kern-.05em{\sc i\kern-.025em b}\kern-.08em T\kern-.1667em\lower.7ex\hbox{E}\kern-.125emX}}

\begin{document}
    % Title
    \title{Human Activity Recognition Using Smartphones\\}

    % Authors
    \author{\textit{Kalos Lazo, Lucas Carranza, Benjamín Soto, José Osnayo}}
    % \author{
    %     \IEEEauthorblockN{Kalos Lazo}
    %     \IEEEauthorblockA{\textit{kalos.lazo@utec.edu.pe}}
    %     \and
    %     \IEEEauthorblockN{Lucas Carranza}
    %     \IEEEauthorblockA{\textit{lucas.carranza@utec.edu.pe}}
    %     \and
    %     \IEEEauthorblockN{Benjamín Soto}
    %     \IEEEauthorblockA{\textit{benjamín.soto@utec.edu.pe}}
    %     \and
    %     \IEEEauthorblockN{José Osnayo}
    %     \IEEEauthorblockA{\textit{jose.osnayo@utec.edu.pe}}
    % }
\maketitle

\begin{abstract}
En el presente documento se discute acerca de la resolución de un problema de clasificación supervisada a partir de un estudio realizado a un grupo de voluntarios de entre $19$ y $48$, mientras realizaban actividades cotidianas: \textit{caminar, subir escaleras, bajar escaleras, sentarse, levantarse y acostarse} se obtuvieron datos del acelómetro y giroscopio embebido sus teléfonos inteligentes: \textit{Samsung Galaxy SII}. Los datos fueron convertidos usando la librería TsFresh para extraer las características de las series temporales más importantes y de manera sencilla. Posteriormente se 
\end{abstract}

\section{Introduction}
Project description

\section{Conjunto de Datos}
Exploration and analysis of the dataset

\section{Metodología}
Explanation of the model, loss functions, and regularization techniques

\section{Implementación}
Include the link to Colab or GitHub where the implementation can be found, avoiding direct
code placement in the report. Define a seed to replicate the results. [Optional] Relevant implementation
details can also be included (error handling, parallelization, etc.).

\section{Experimentación}
Present results with graphs and/or tables, avoiding terminal screenshots

\section{Discusión}
Interpretation of the obtained results and their relationship with the learned theory

\section{Conclusiones}
Summary of results, limitations, and recommendations

\end{document}
